\section{Materials}\label{sec:Materials}
\todo[inline, color=yellow]{Laura}
The following sections describe the resources and tools required for the completion of the project. \\ Furthermore, the test images are presented in chapter \ref{sec:testimages}.


\subsection{Hardware}\label{sec:Hardware}
\todo[inline, color=yellow]{Laura}
During the implementation phase, the application was run on two computers, which are described in the following two sections. Both computers needed to be able to deal with the software components described in section \ref{sec:Software}. An extract from your data sheet is shown in table \ref{tab:Computer1} respectively table \ref{tab:Computer2}.


\begin{table}
	\centering
	\begin{tabular}{|l|l|}
		\hline
		\Absatzbox{}
		\textbf{\textcolor{red}{NAME?}}& \textbf{Description} \\
		\hline
		Processor & \textcolor{red}{??} \\
		\hline
		RAM & \textcolor{red}{??}  \\
 		\hline 
		Graphic Card & \textcolor{red}{??} \\
		\hline
		Operating System & \textcolor{red}{??}  \\
		\hline
	\end{tabular}
	\caption[Extract from the Data Sheet of the \textcolor{red}{NAME?}]{Extract from the Data Sheet of the \textcolor{red}{NAME?}}.
	\label{tab:Computer1}
\end{table}

\begin{table}
	\centering
	\begin{tabular}{|l|l|}
		\hline
		\Absatzbox{}
		\textbf{Acer Aspire 5820TG}& \textbf{Description} \\
		\hline
		Processor & Intel Core i3 CPU @ $2.40\,$GHz \\
		\hline
		RAM & $4\,$GB \\
 		\hline 
		Graphic Card 1 & AMD Mobilty Radeon HD 5000 Series\\
		\hline
		Graphic Card 2 & Intel(R) HD Graphics\\
		\hline
		Operating System & Windows 10 Education 64 bit \\
		\hline
	\end{tabular}
	\caption[Extract from the Data Sheet of the Acer Aspire 5820TG Notebook.]{Extract from the Data Sheet of the Acer Aspire 5820TG Notebook.}
	\label{tab:Computer2}
\end{table}


\subsection{Software} \label{sec:Software}
\todo[inline, color=yellow]{Laura}
In order to develop the \textit{Interactive Lighting Detector} \textit{Qt} was used (compare section \ref{sec:qt}). To take advantage of already existing functionalities the \textit{OpenCV}-library, which is described in section \ref{sec:opencv}, was taken advantage of.
\subsubsection{QT} \label{sec:qt}
\todo[inline, color=red]{Laura}

\subsubsection{OpenCV} \label{sec:opencv}
\todo[inline, color=red]{Laura}
The \textit{Open Source Computer Vision} (OpenCV) is an open source library for image- and video processing, which is among others available in the programming language \textit{C}$++$. It has been introduced ten years ago and is developed by various programmers since then. This library offers the most common algorithms, as well as current developments in image processing \cite{article:OpenCV}.\\
\textcolor{red}{Für dieses System ist vor allem das Modul \texttt{calib3d}~\cite{website:Calib3dDoc} und das extra Modul \texttt{aruco}~\cite{website:ArucoDoc} verwendet. Das erste Modul \texttt{calib3d}  bietet alle notwendigen Funktionen zur Erstellung, Verwendung und Weiterverarbeitung von intrinsischen und extrinsischen Kamerakalibrierungen an (vgl. Abschnitt~\ref{sec:calib}). Während das Zweite alle benötigten Ressourcen und Funktionalitäten zum Tracking von \textit{ArUco} Markern zur Verfügung stellt (vgl. Abschnitt~\ref{sec:aruco}).}






\subsection{Testimages} \label{sec:testimages}
\todo[inline, color=yellow]{Laura}
Due to the assumption that the objects shown on the test images described in section~\ref{sec:testimagesfirst} have a too complicated shape, a second batch of images was made (compare section~\ref{sec:testimagessecond}). Images of both batches were used to test the functionality of the the algorithms used for the lighting detection. All images have in common that besides the actual object they show a sundial to simplify the determination of the light direction for the user.


\subsubsection{First Batch} \label{sec:testimagesfirst}

For examples of the first batch of test images are shown on figure \ref{fig:batch1}. Next to the mandatory sundial there are different objects depicted.

\todo[inline, color=red]{Laura}
\begin{figure}[H] 
	\center 
	\includegraphics[width=12cm]{Images/batch1.jpg}			
	\caption[Examples of the Test Images of the first Batch.]{Examples of the Test Images of the first Batch.}
	\label{fig:batch1}
\end{figure}

\subsubsection{Second Batch} \label{sec:testimagessecond}


\todo[inline, color=red]{Laura}
\begin{figure}[H] 
	\center 
	\includegraphics[width=12cm]{Images/batch2.jpg}			
	\caption[Examples of the Test Images of the second Batch.]{Examples of the Test Images of the second Batch.}
	\label{fig:batch2}
\end{figure}

\newpage