\section{Introduction}\label{sec:Introduction}
The detection of digital image forgery is an actual research topic, which analyses various image properties to get a conclusion about the authenticity of a digital image. It can be divided into two sections~\cite{4284575}. The first section is the source digital camera identification. It considers image properties like lens aberrations, sensor imperfections or use a colour filter array (CFA) interpolation to analyse the suspicious items. Those mentioned properties can give an assumption on the camera or even the camera model, which was used to capture the picture.

The other section is the actual image forgery analysis. A common method to detect the application or device that was used to save the image file is the analysis of JPEG patterns. Each has an individual JPEG pattern which can be extracted and compared to a database of acknowledged patterns. Another approach for image forgery analysis is the detailed observation of the chromatic aberration camera response function (CRF) that can give conclusions about manipulations in form of retouching or composing of new image content. 

Due to the assumption that different images are taken under different light conditions, the manipulative composition of two images can be often detected by analysing the light vectors of various objects depicted in the image. But this method is also very difficult because it needs an image sequence of the scene to estimate the 3-dimensional light direction successfully with established algorithms. This is not very useful in reality because in most cases only the suspect image is available without any information about its real environment. Nevertheless, there exists an approach by Johnston and Farid~\cite{Johnson} that can rather estimate the object's light direction under some pre defined assumptions even from single images with a common least square approach. The assumptions are clearly defined and result in a simplified light model that is related to an infinite light source and a constant reflection values of the object's surface \textcolor{red}{Stimmt das so? Der Refelektionswert wird ja für jeden patch als konstant angenommen und nicht für das ganze Objekt. Würde da nur das mit der unendlichen Lichtquelle schreiben, weil sich der rest pro approach unterscheidet :)}.

\subsection{Motivation}\label{sec:Motivation}
In the last decade, it became more easily to create an image forgery for everyone. Software applications like \textit{Adobe Photoshop} or \textit{Gimp} and almost every mobile phone or social media platform offers possibilities to sophisticate digital images. Every user can manipulate images or compose a new image of various other images without any significant knowledge. For example, two portraits of prominent people can be combined in a way that the impression of a relationship accrues which does not really exist. If this is made in an imperfect way the forgeries can be recognised very quickly by the human visual system but if they are made in a professional way it is quite challenging to detect the manipulation. 

In this case, a professional analysis tool is required to give a reliable rating which can stand in court as evidence. As mentioned, a main problem of the analysis is that many approaches ask for a sequence of images from the scene for trustworthy and validated results but in most cases only the suspect image is available. Thus, an approach which estimates the light vector under some refused assumption has to be used and eventually confirmed by other image forgery detection methods. 

\subsection{Project Goal}\label{sec:Project Goal}
According to the previous sections, the project goal is to implement and evaluate the approach of Johnston~\cite{Johnson} in \textit{C++} with an adequate \textit{Qt} Graphic User Interface (GUI). This GUI offers a segmentation of the object of interest, as well as an interactive selection of the most suitable contour part, to estimate the light direction. The latter is an important requirement of the Johnston approach. To validate the results, several test images have to be captured with a simulated infinite light source. The infinite light source can be the sun on a cloudless day, for example. A sun clock needs to be added to the test scene to verify the current light direction and various objects with a simple and even shape complete it. All resulting vectors and contours are printed in the actual image or respectively saved in a text file for the following evaluation.



\newpage


















